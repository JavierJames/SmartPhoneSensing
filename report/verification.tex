\chapter{Verification Results}	\label{chap:verification}
	In order to test whether the model satisfies the requirements specified in \cref{chap:requirements} the accompanying $\mu$-calculus formulas in \cref{chap:tests} have to be tested.
	In this chapter a detailed description of how the verification process was carried out is given.
	
	The verification was done using the mCRL2 toolset~\cite{mcrl2website}. 
	Tools such as \texttt{mcrl22lps} , \texttt{lps2pbes} and \texttt{pbes2bool} were used in this process.
	The version of the mCRL2 toolset used is \texttt{201310.0 (Release)}.
	
	The \texttt{mcrl22lps} tool takes the process specification from our model as input and converts it to a Linear Process System, see \cref{mcrl22lps}. 
	
	\begin{lstlisting}[caption=Transformation of mCRL2 model to LPS description, label=mcrl22lps ]
	mcrl22lps model.mcrl2 model.lps
	\end{lstlisting}
	
	With the model in {\em lps-format} the tests can be run with command
	\cref{ver_cmd} where {\em <requirement>} is the name of	the requirement at hand, for example {\em SR3}.
	The tool \texttt{lps2pbes} was used to create a parameterised boolean equation system (PBES) from the LPS description of our model and the requirement at hand.
	The result is piped to \texttt{pbes2bool} to retrieve
	the boolean value which shows whether the test passes or not.
	 
	\begin{lstlisting}[caption=Requirement testing ,label=ver_cmd] 
	lps2pbes model.lps -f <requirement>.mcf | pbes2bool
	\end{lstlisting}

	There are many tests to run and  each test takes a minute or two to compute. To speed up the process the tests were run in parallel using GNU Parallel\cite{Tange2011a}. The \cref{ver_cmd_parallel} shows
	the command used to speed up the verification with at least a factor four.
	\begin{lstlisting}[caption=Parallelization of verification command, label=ver_cmd_parallel]
	parallel -k --tag 'lps2pbes model.lps -f {} | pbes2bool' ::: *.mcf
	\end{lstlisting}
	
	\cref{tbl:verificationResults} depicts the results of running all 13
	tests listed in \cref{chap:tests} .
	The results show that the model passed all the functional and safety requirements.
	This indicates that the model can be considered safe.
		
	\begin{table}[!htbp]
		\begin{center}
		\begin{tabular}{l | r r}
			Requirement type & \# passed & \# failed \\ \hline
			Functional & 3 & 0 \\
			Safety & 10 & 0	\\
		\end{tabular}
		\caption{Model verification results}
		\label{tbl:verificationResults}
		\end{center}
	\end{table}

	
	\cref{tbl:trivialResults} depicts the results of running all properties for the trivial paths and the existential tests. 
	The results show that all trivial paths exist for the first five requirements.
	It can also be seen that all existential tests pass, so all paths that should exist according to our safety requirements do actually exist.
	Based on these results we can conclude that the safety requirements do not hold vacuously.
		
	\begin{table}[!htbp]
		\begin{center}
		\begin{tabular}{l | r r}
			Property type & \# passed & \# failed \\ \hline
			Trivial paths & 23 & 0	\\
			Existential tests	&	10	&	0	\\
		\end{tabular}
		\caption{The verification results of the trivial paths and existential tests}
		\label{tbl:trivialResults}
		\end{center}
	\end{table}
	
	\cref{tbl:sensorResults} depicts the results of running all properties for the sensors.
	The results show that the implementation of the sensors passes all requirements
	
	\begin{table}[!htbp]
		\begin{center}
		\begin{tabular}{l | r r}
		Property type	& \# passed & \# failed \\	\hline
		Sensor			& 3	& 0 \\
		Existential tests & 3 & 0 \\
		\end{tabular}
		\caption{The verification results for the sensors}
		\label{tbl:sensorResults}
		\end{center}
	\end{table}
	   
   
