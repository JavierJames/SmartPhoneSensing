\chapter{Discussion \& Conclusions}	\label{chap: conclusion}
	In this project we aimed to design a safety controller which takes commands from the bridge operator 
	and presents them to the hardware, while taking in mind security of bridge operation.
	It was essential that this happens in a safe way, to prevent future accidents from happening.
	
	In order to build our system, we made some assumptions to simplify the design of the safety controller.
	One important assumption was that the user does not need to control the motor,
	brakes and deck independently. % I'm not sure if this is the 'most important one'
	This is done by a grouping interactions into one single command for the user.
	We designed a model which consists of four components: safety controller, hardware controller, status controller, and sensors.  
	
	We verified the safety of the model with functional and safely requirement using the MCRL2 tools.
	The model passed all tests and it is therefore considered to meet the requirements and is considered safe.
	Even though the model passed all test, it should be clear that this does not mean that this design can be used in the real world.
	Before this can happen, it still requires more strict safety requirements to ensure that the system does not fail.
	However, this model still demonstrates the beginnings of a bridge controller that adheres to safety requirements. 
