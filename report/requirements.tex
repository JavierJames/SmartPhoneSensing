\chapter{Model Requirements}\label{chap:requirements}
	As mentioned before, this safety controller should provide a safe environment in which bridge operators can control the bridge. 
	In order for us to ensure this, we began by specifying requirements to verify the safety and functionality of our design. 
	In this chapter we will first discuss the important model decisions that have been made, based on which our requirements were formulated. 
	Subsequently, we will provide the system global requirements. 
	  
	In our design we assume that there is an hardware mechanism present that ensures that the motor does not open/close the bridge beyond a maximum point. 
	This entails that if the user tries to open the bridge when it is already open, it will not be possible due to the hardware.
	Our safety controller is not responsible for resolving this error.
	
	There are two types of actions possible in the system. 
	The first type follows directly from a user command. 
	The second type of actions follow on the first type and are therefore indirectly related to a user command. 
	This means that the system is not allowed to perform actions that are completely initiated by itself.
	
	In the system it is allowed that one pre-sign and/or one stop sign is damaged. 
	Therefore such failures should not directly block the operation of the bridge. 
	If there are more failures the operation of the bridge should stop, because the system is then considered to be unsafe.


	\section{Functional Requirements}
		In order to ensure a proper operation of the safety controller for the bridge, we needed to define the specific behaviour of our system. 
		This was done by modelling the system behaviour with functional requirements.  
	 	
	 	
	 	 
		%First of all some functional requirements have to be defined to ensure a proper operation of the bridge.
		\begin{enumerate}
			% FR1
			\item The system should not contain any deadlocks.
			
			% FR2
			\item Cars should always have the possibility to safely pass the bridge.
			
			% FR3
			\item Boats should always have the possibility to safely pass the bridge.
		\end{enumerate}
	
	\section{Safety Requirements}
		To guarantee the safety of boats and cars, it was essential to define the specific system behaviour needed to make this possible. 
		We did this by providing the following safety requirements.
	
		\begin{enumerate}
			% SR1
			\item The stop signs should only be turned on when the pre signs are on.
			
			% SR2
			\item The barriers should only be closed when the stop-signs are on.
			
			% SR3
			\item The locking pins should only deactivate when the barriers are closed.
			
			% SR4
			\item The brakes should only be turned off when the locking pins are deactivated.
			
			% SR5
			\item The motor should only move upward when the brakes are disabled.
			
			% SR6
			\item The motor should only be stopped when the brakes is engaged.
			
			%SR7
				
			\item The locking pins should only activate when the motor has moved the bridge deck down.
		
			% SR8
			\item The barriers can only be opened when the locking pins are enabled.
			
			% SR9
			\item The stop signs should only be turned off when the barriers are opened.
			
			% SR10
			\item The pre-signs should only be turned off when the stop signs are off.
		\end{enumerate}
